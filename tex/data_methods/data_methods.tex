\documentclass{article}

\title{Data appendix}
\author{}
\date{}

\usepackage{hyperref}
\usepackage{breakurl}
\usepackage{amsmath}
\usepackage{cite}

\begin{document}
\maketitle

\tableofcontents

\section{Classification of workers by industry and/or occupation}
For mapping Census occupation and industry codes to SOC and NAICS codes, respectively, I rely on crosswalks provided by \cite{crosswalks}.

\subsection{Occupation classification}
We use a total of 94 occupation categories, consisting of all three-digit (aka minor) groups in the 2010 SOC system, with the exclusion of major group 55 known as \emph{Military Specific Occupations}. For cases in which the data are based on other classification systems, e.g. 2018 SOC or 2010 Census, we use crosswalks to reclassify workers into a 2010 minor group.

\subsection{Sector classification}
To identify a worker as belonging to sector C or sector S, we go by 2017 NAICS codes. For datasets using 2012 NAICS codes, we map 2012 industry codes into roughtly equivalent 2017 industry codes.

\section{BLS Occupational Employment Statistics}
\subsection{Data}
We use industry-specific, cross-ownership OES data downloaded from the BLS website (See \cite{OES}). For most years, separate datasets are available for different levels of industry aggregation. Depending on the context, we use the broadest level of industry aggregation possible, to reduce the likelihood of our estimates being affected by rounding error or missing values present at very fine levels of industry aggregation. 

\subsection{Aggregating occupation to 2010 SOC minor groups}
\label{OES: aggregation}
We use OES data on employment and mean wage by occupation and industry.
Depending on the year and occupation in question, OES datasets in some cases provide employment and mean wage at multiple levels of occupation aggregation, and sometimes our desired minor groups are omitted.
When estimating employment for each of our occupation categories, we first use the employment figures provided at the 2010 SOC minor level, if available. Where they are not provided, we aggregate from finer occupation groups. We do the same for the mean wage, except we take employment-weighted means as we aggregate up rather than sums.

\subsection{Aggregating by both industry and occupation}
In some cases we aggregate by both industry and occupation. We do this by separately identifying each worker with a minor 2010 SOC code and a code for industry, with the latter depending on our desired level of industry aggregation for a given estimator. Then we aggregate up as described in Section \ref{OES: aggregation}, with the change that we compute estimates conditional on both occupation and industry rather than occupation alone.

\section{ACS}
Our source for the ACS is \cite{ACS}, provided by \cite{IPUMS}.

\subsection{Sample}
We use an individual-level sample restricted to observations satisfying the following:
\begin{itemize}
\item At least 15 years of age
\item In the labor force
\item Annual earnings last year were at least \$1000
\item Worked at least 27 weeks of the previous year
\end{itemize}

\subsection{Variables}
\subsubsection{Weights}
For weighting, we use the PERWT variable in all estimation.

\subsubsection{WFH}
We classify an individual as having worked from home based on the response to the question: ``How did this person usually get to work LAST WEEK?''
Each observation was classified as WFH if the response was ``Worked at home,'' otherwise the observation was classified as not a WFH worker.
We use the ACS variable called TRANWORK.

\section{ATUS}

\subsection{Sample}
Our sample consists of respondents to the 2017-2018 ATUS Leave Module. This module by construction excludes all non-workers and all self-employed workers. I do not impose any sample restrictions.

\subsection{Variables}

\subsubsection{Weights}
All estimates are weighted using the ATUS variable \emph{lufinlwgt}.

\subsubsection{WFH}
We base our WFH statistics on the variables \emph{lejf\_11} and \emph{lujf\_10}, which correspond to actual WFH and the ability to WFH in one's job, respectively.

\section{SIPP}
\subsection{Sample}
We use waves 1-4 of the 2014 SIPP. For each wave, we use all individuals of age 15 or older who were present in all 12 months of the wave. The waves are treated separately when computing all intermediate variables, and then pooled prior to final estimation. We do not link individuals across waves in any way.

\subsection{Classification of worker into occupation category}
\label{sipp_occupation}
Respondents to SIPP are allowed to provide up to seven distinct jobs, each of with is reported for the individual months out of the year during which the individual held said occupation.
For each worker-wave pair, I classify a worker as holding a given occupation by identifying one of the reported occupations as the primary occupation. I enact the following procedure for each worker-wave pair:
\begin{enumerate}
\item I make a list of all distinct occupation codes reported.
\item I compute the number of months for which each distinct occupation code is reported. If the code is repeated for multiple reported jobs in the same month, I count this just once.
\item I designate the primary occupation as the occupation code reported in the greatest number of months. If there is a tie, I choose among the most frequent codes the occupation code which was reported first.
\end{enumerate}

\subsection{Classification of worker into industry category}
For industry classification I use essentially the same procedure as in Section \ref{sipp_occupation}, and I do this separately from occupation.

\subsection{Variables}

\subsubsection{Weights}
All estimates are computed based on the provided person-level weight variable for the 2014 SIPP, which is identified as WPFINWGT.

\subsubsection{Assets and liabilities}
We use the following definitions:
\begin{align*}
\text{deposits} &= \text{saving accts} + \text{checking accts} + \text{money market funds} \\
\text{bonds} &= \text{gov bonds} + \text{municipal and corporate bonds} \\
\text{liquid assets} &= 1.05 \times (\text{deposits} + \text{bonds} + \text{stocks} + \text{mutual funds}) \\
\text{net liquid assets} &= \text{liquid assets} - \text{cc debt} \\
\text{net illiquid assets} &= \text{home equity} + \text{IRA} + \text{Keogh} + \text{CDs} + \text{life insurance}
\end{align*}
Home equity is surveyed at the household-level but SIPP provides an individual-level recode. The rest of the above variables are provided at the individual level.

\subsubsection{Earnings}
Our measure of earnings is annual earnings, which we set equal to the sum of TPEARN over all twelve months of the year. We take weekly earnings to be annual earnings divided by 52. The description of this variable, taken from \cite{tpearn}, is the following:
\begin{quote}
Income earned from all jobs worked during the month, including wage and salary income, bonus payments, commissions, overtime payments, tips, other income from self-employed businesses, self-employed business profits, and accounting for time spent away from a job without pay.
\end{quote}

\subsubsection{Ratio of liquid wealth to earnings}
This statistic is just net liquid assets divided by annual earnings, conditional on annual earnings being at least \$1000.

\subsubsection{HtM in terms of earnings}
For various $x$ and $y$ in the expressions below, we compute binary indicators for HtM status for each observation. Let $b$ be net liquid assets and let $a$ be net illiquid assets.
\begin{align*}
\text{HtM} &= (a < x) \\
\text{WHtM} &= (a < x) * (b \geq y) \\
\text{PHtM} &= (a < x) * (b < y)
\end{align*}
To compute shares of individuals falling into these categories, we take weighted means of these indicators.

\section{Essential workers by occupation}
\subsection{Data}
Here we use a combination of data from \cite{OES} and \cite{brookings}. The appendix to \cite{brookings} provides 4-digit industry codes deemed to be essential. The OES sample used was the four-digit employment-by-industry dataset for May 2017.

\subsection{Methods}
First, each OES occupation code was mapped to a 2010 SOC minor category. Second, each four-digit industry was either mapped to the essential workers data, in which case the industry was deemed essential, or not, and the industry was classified as non-essential.
At this point we had employment, mean wage, and a binary indicator for essential industries, at the occupation-industry level. We then computed employment-weighted means to aggregate over all industries and estimate the share of each occupation in essential industries.


\section{A teleworkable index using data from Dingel and Neiman}

\subsection{Data}
Using a binary indicator of whether an occupation is \emph{teleworkable} from \cite{DN}, we aggregate up to 2010 SOC minor occupation categories.

\subsection{Methods}
Note that for some six-digit occupations, the teleworkable indicator was not provided. In these cases, the occupations with a missing indicator were ignored and we proceeded with aggregation.

We aggregated upwards starting from the finest categories possible. We took employment-weighted averages when each of the occupations listed in Dingel-Neiman had a non-missing employment estimate in OES for the given sector. Elsewhere, we took arithmetic means. 

This means that, for example, once we averaged out the SOC-O*NET occupations to 6-digit SOC codes, we might run into a case where two of the three 6-digit occupations in Dingel-Neiman for a given 5-digit occupation category have nonmissing employment estimates in OES, but one is missing in OES. In that case, we took another arithmetic mean over the teleworkable variable for these three occupations to aggregate to the 5-digit level. From there, we would then do the same thing: if all occupations in SOC at the given 5-digit category had nonmissing employment estimates in OES, we would use an employment-weighted mean of teleworkable at the current aggregation level to aggregate upward, and if not, we would use an arithmetic mean, etc...


\bibliography{bibliography}
\bibliographystyle{plain}

\end{document}