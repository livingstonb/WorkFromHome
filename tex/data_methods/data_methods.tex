\documentclass{article}

\title{Data appendix}
\author{}
\date{}

\usepackage{booktabs}
\usepackage{hyperref}
\usepackage{breakurl}
\usepackage{amsmath}
\usepackage{cite}
\usepackage{graphicx}
\graphicspath{ {./images/} }

\begin{document}
\maketitle

\tableofcontents

\section{Classification of workers by industry and/or occupation}
For mapping Census occupation and industry codes to SOC and NAICS codes, respectively, I rely on crosswalks provided by \cite{crosswalks}.

\subsection{Occupation classification}

\subsubsection{Minor groups}
We use a total of 94 occupation categories, consisting of all three-digit (aka minor) groups in the 2010 SOC system, with the exclusion of major group 55 known as \emph{Military Specific Occupations}. For cases in which the data are based on other classification systems, e.g. 2018 SOC or 2010 Census, we use crosswalks to reclassify workers into a 2010 minor group.

\subsubsection{Broad groups}
We use a total of 443 occupation categories, consisting almost entirely of five-digit (aka broad) groups in the 2010 SOC system, with the exclusion of occupations in major group 55 known as \emph{Military Specific Occupations}.
All five-digit groups are included except for those in four three-digit groups for which we aggregate to the three-digit level. These three-digit groups are the following:
\begin{itemize}
\item Postsecondary Teachers
\item Other Teachers and Instructors
\item Other Healthcare Practitioners and Technical Occupations
\item Supervisors of Transportation and Material Moving Workers
\end{itemize}

\subsection{Sector classification}
To identify a worker as belonging to sector C or sector S, we go by 2017 NAICS codes. For datasets using 2012 NAICS codes, we map 2012 industry codes into roughtly equivalent 2017 industry codes.

\section{BLS Occupational Employment Statistics}
\subsection{Data}
We use industry-specific, cross-ownership OES data downloaded from the BLS website (See \cite{OES}). For most years, separate datasets are available for different levels of industry aggregation. Depending on the context, we use the broadest level of industry aggregation possible, to reduce the likelihood of our estimates being affected by rounding error or missing values present at very fine levels of industry aggregation. 

\subsection{Aggregating occupation to 2010 SOC minor groups}
\label{OES: aggregation}
We use OES data on employment and mean wage by occupation and industry.
Depending on the year and occupation in question, OES datasets in some cases provide employment and mean wage at multiple levels of occupation aggregation, and sometimes our desired minor groups are omitted.
When estimating employment for each of our occupation categories, we first use the employment figures provided at the 2010 SOC minor level, if available. Where they are not provided, we aggregate from finer occupation groups. We do the same for the mean wage, except we take employment-weighted means as we aggregate up rather than sums.

\subsection{Aggregating by both industry and occupation}
In some cases we aggregate by both industry and occupation. We do this by separately identifying each worker with a minor 2010 SOC code and a code for industry, with the latter depending on our desired level of industry aggregation for a given estimator. Then we aggregate up as described in Section \ref{OES: aggregation}, with the change that we compute estimates conditional on both occupation and industry rather than occupation alone.

\section{ACS}
Our source for the ACS is \cite{ACS}, provided by \cite{IPUMS}.

\subsection{Sample}
We use an individual-level sample restricted to observations satisfying the following:
\begin{itemize}
\item At least 15 years of age
\item In the labor force
\item Annual earnings last year were at least \$1000
\item Worked at least 27 weeks of the previous year
\end{itemize}

\subsubsection{Sample size}
For each year between 2013 and 2017 we have roughly 1.2 million observations with a non-missing response to the WFH question, after making the restrictions above and dropping workers in occupations or industries outside of our classifications:

\begin{table}[ht]
\caption{ACS sample sizes}
\centering
\begin{tabular}{c | c}
\hline%inserts double horizontal lines
Survey Year & n \\
%heading
\hline % inserts single horizontal line
2013 & 1,140,183  \\ % inserting body of the table
2014 & 1,149,987 \\
2015 & 1,172,018 \\
2016 & 1,189,541 \\
2017 & 1,217,299 \\ [1ex] % [1ex] adds vertical space
\end{tabular}
\label{table:ACS sample size} % is used to refer this table in the text
\end{table}

\subsection{Variables}
\subsubsection{Weights}
For weighting, we use the PERWT variable in all estimation.

\subsubsection{WFH Question}
We classify an individual as having worked from home based on the response to the question:
\begin{quote}
How did this person usually get to work LAST WEEK?
\end{quote}
Each observation was classified as WFH if the response was:
\begin{quote}
Worked at home
\end{quote}
Otherwise the observation was classified as not a WFH worker.
These responses are provided in the variable TRANWORK.
See Figure \ref{ACS-WFH} for an example from the 2017 questionnaire.\footnote{\url{https://www2.census.gov/programs-surveys/acs/methodology/questionnaires/2017/quest17.pdf}}

\begin{figure}
\centering
\label{ACS-WFH}
\includegraphics[width=0.5\linewidth]{ACS_workfromhome.png}
\caption{WFH in the 2017 ACS questionnaire}
\end{figure}

\section{ATUS}

\subsection{Sample}
Our sample consists of respondents to the 2017-2018 ATUS Leave Module. This module by construction excludes all non-workers and all self-employed workers. I do not impose any sample restrictions.

\subsubsection{Sample size}
Our final sample size after dropping workers outside of our industry or sector classifications or for whom WFH answers were missing is 9,456.

\subsection{Variables}

\subsubsection{Weights}
All estimates are weighted using the ATUS Leave Module variable LUFINLWGT.

\subsubsection{WFH Question}
We base our WFH statistics on the variables LUJF\_10 and LEJF\_11, which correspond to the ability to WFH in one's job, and actual WFH in one's job, respectively. These questions are as follows, with possible responses of ``Yes,'' ``No,'' and ``Don't Know'', the last of which I discard when estimating means over the given variable:
\begin{quote}
JF\_10: \\
As part of your (main) job, can you work at home?
\end{quote}
\begin{quote}
JF\_11: \\
Do you ever work at home?
\end{quote}

The prefixes LE and LU denote an edited variable and an unedited variable, respectively. ATUS provides one question as edited and the other as unedited, hence the different prefixes. Since the universe of LEJF\_11 is restricted to observations with a ``Yes'' response to LUJF\_10, we recoded LEJF\_11 to imply no work from home if the value of LUJF\_10 implied being unable to work from home.

\section{SIPP}
\subsection{Sample}
We use waves 1-4 of the 2014 SIPP. The waves are treated separately when computing all intermediate variables, and then pooled prior to final estimation. We do not link individuals across waves in any way.

\subsubsection{Individual level sample}
In the individual level sample, we use all individuals of age 15 or older who were present in all 12 months of the wave.

\subsubsection{Family and household level samples}
Here we start with all individuals of age 15 or older who were present in all 12 months of the wave. SIPP identifies families/households for each month separately, and because of changes in family/household composition across months, we use our own definition of family/household at the annual frequency. We define a sampling unit with the following procedure, repeated separately for each wave:

\begin{enumerate}
\item Designate as a head for each SIPP-designated family (household) reference member who was a family (household) reference member for each month of the wave. For families, the heads are the observations satisfying PNUM = RFAMREF for each value of MONTHCODE between 1 and 12. For households, the heads are the observations satisfying PNUM = RFAMREF and RFAMNUM = 1 for each value of MONTHCODE between 1 and 12.
\item Create a one-member sampling unit for each head.
\item For each sampling unit, check if the head had a spouse present at any month in the wave. If a spouse was not present (EPNSPOUS\_EHC was missing for all months for the head), or if the head's marriage to the spouse was constant (EPNSPOUS\_EHC was a constant across all twelve values of MONTHCODE) do nothing. Otherwise drop the sampling unit and drop both the head and the spouse from the dataset. This drops only a small number of families/households.
\item For each individual who is not a sampling unit head, check if the individual's family/household reference member was the same individual for all twelve values of MONTHCODE. If so, add the individual to the sampling unit created for that reference member, if that reference member is still in the sample. If the reference member changed for the individual or if the reference member was constant but has been dropped from the dataset, drop the individual from the dataset. I.e. a member of a family/household will only be included in the sampling unit if that member is present in that particular family/household for the entire wave.

An individual's household reference member is the observation with the same values for SSUID and ERESIDENCEID as the individual, in the given month and wave, and who as already been identified as a household reference member in the first step. The same is true for identifying the individual's family reference member, except that the individual and the family reference member must also share the same value of RFAMNUM.
\end{enumerate}

\subsubsection{Sample size}
For computing \%WFH by occupation, we use 98,597 observations across four waves, with each observation representing one respondent-year pair. Since the source data is a panel, this amounts to 42,170 distinct individuals. These figures reflect both attrition and new entrants into existing SIPP sample units. 

\subsection{Classification of worker into occupation category}
\label{sipp_occupation}
Respondents to SIPP are allowed to provide up to seven distinct jobs, each of with is reported for the individual months out of the year during which the individual held said occupation.
For each worker-wave pair, I classify a worker as holding a given occupation by identifying one of the reported occupations as the primary occupation. I enact the following procedure for each worker-wave pair:
\begin{enumerate}
\item I make a list of all distinct occupation codes reported.
\item I compute the number of months for which each distinct occupation code is reported. If the code is repeated for multiple reported jobs in the same month, I count this just once.
\item I designate the primary occupation as the occupation code reported in the greatest number of months. If there is a tie, I choose among the most frequent codes the occupation code which was reported first.
\end{enumerate}

\subsubsection{Household and family level samples}
In the household and family level samples, the occupation of the household or family is set to the occupation of the individual in the household or family with the largest reported annual earnings.

\subsection{Classification of worker into industry category}
For industry classification I use essentially the same procedure as in Section \ref{sipp_occupation}, and I do this separately from occupation.

\subsubsection{Household and family level samples}
In the household and family level samples, the industry of the household or family is set to the industry of the individual in the household or family with the largest reported annual earnings.

\subsection{Variables}

\subsubsection{Weights}
The weights used were chosen according to a flow diagram provided for the 2014 SIPP.

Individual-level estimates are computed based on the December value of the provided person-level weight variable for the 2014 SIPP, which is WPFINWGT.

Family-level estimates are computed based on the December value of WPFINWGT for the family reference member.

Household-level estimates are computed based on the December value of WPFINWGT for the household member for which ERELRPE is equal to either 1 or 2.


\subsubsection{Assets and liabilities}
At the individual level, the individual level variables below are used directly. At the household and family level, an unweighted sum is taken over these variables across members of the household or family.
We use the following definitions:
\begin{align*}
\text{deposits} &= \text{saving accts} + \text{checking accts} + \text{money market funds} \\
\text{bonds} &= \text{gov bonds} + \text{municipal and corporate bonds} \\
\text{liquid assets} &= 1.05 \times (\text{deposits} + \text{bonds} + \text{stocks} + \text{mutual funds}) \\
\text{net liquid assets} &= \text{liquid assets} - \text{cc debt} \\
\text{net illiquid assets} &= \text{home equity} + \text{IRA} + \text{Keogh} + \text{CDs} + \text{life insurance}
\end{align*}
Home equity is surveyed at the household-level but SIPP provides an individual-level recode. The rest of the above variables are provided at the individual level.

\newcommand{\eqindent}{\\ \indent \hspace{4pt} = }
\newcommand{\listassets}[1]{sum(TJS{#1}VAL TJO{#1}VAL TO{#1}VAL)}

The variables we use are as follows:
\begin{itemize}
\item Saving accounts \eqindent \listassets{SAV}
\item Interest-bearing checking accounts\eqindent  \listassets{ICHK}
\item Non-interest-bearing checking accounts\eqindent  \listassets{CHK}
\item Money market funds\eqindent  \listassets{MM}
\item Government bonds\eqindent  \listassets{GOVS}
\item Municipal and corporate bonds\eqindent  \listassets{MCBD}
\item Stocks\eqindent  \listassets{ST}
\item Mutual funds\eqindent  \listassets{MF}
\item Credit card debt = TDEBT\_CC
\item Home equity = TEQ\_HOME
\item IRA and Keogh = TIRAKEOVAL
\item CDs = \listassets{CD}
\item Life insurance = TLIFE\_CVAL
\end{itemize}

\subsubsection{Earnings}
Our measure of earnings is annual earnings, which we set equal to the sum of TPEARN over all twelve months of the year. We take weekly earnings to be annual earnings divided by 52. The description of this variable, taken from \cite{tpearn}, is the following:
\begin{quote}
Income earned from all jobs worked during the month, including wage and salary income, bonus payments, commissions, overtime payments, tips, other income from self-employed businesses, self-employed business profits, and accounting for time spent away from a job without pay.
\end{quote}

At the individual level, TPEARN is used directly. At the household or family level, an unweighted sum is taken over TPEARN across members of the household or family.

\subsubsection{Ratio of liquid wealth to earnings}
This statistic is just net liquid assets divided by annual earnings, conditional on annual earnings being at least \$1000.

\subsubsection{HtM in terms of earnings}
For various $x$ and $y$ in the expressions below, we compute binary indicators for HtM status for each observation. Let $b$ be net liquid assets and let $a$ be net illiquid assets.
\begin{align*}
\text{HtM} &= (b < x) \\
\text{WHtM} &= (b < x) * (a \geq y) \\
\text{PHtM} &= (b < x) * (a < y)
\end{align*}
To compute shares of individuals falling into these categories, we take weighted means of these indicators.

\subsubsection{WFH Question}
The variables we use for WFH status in SIPP are labeled EJB\#\_WSHMWRK,
where the \# represents the number of the reported job.
Each of these variables is reported for each month separately, and is reported with ``Yes'' or ``No'' responses.
The question is worded as follows\footnote{This comes from direct correspondence with the SIPP Survey Team.}:
\begin{quote}
As part of (your/name) typical work schedule for (employer name), (are/were) there any days when (you/name) (work, works, worked) only at home?
\end{quote}

\if The codebook describes this variable as the following:
\begin{quote}
Were there any days when \ldots worked only from home?
\end{quote}
and provides additional notes:
\begin{quote}
Respondents who reported days working only at home or reported working from home in combination with another form of transportation to work.
\end{quote}
\fi
To aggregate to the annual frequency, I code an individual as having worked from home if EJB\#\_WSHMWRK contains a ``Yes'' response in any job for that individual, within any of the 12 months of the year. If all responses were ``No'' then the individual is assumed to not have worked from home.
Note that this variable is based on all jobs held; some jobs may not coincide with what we designate as the primary occupation for a given worker. I did (rather quickly) look at the \%WFH-by-occupation figures when defining WFH based only on the values of EJB\#\_WSHMWR coinciding with the designated primary occupation. I didn't see large differences.

\section{Essential workers by occupation}
\subsection{Data}
Here we use a combination of data from \cite{OES} and \cite{brookings}. The appendix to \cite{brookings} provides 4-digit industry codes deemed to be essential. The OES sample used was the four-digit employment-by-industry dataset for May 2017.

\subsection{Methods}
First, each OES occupation code was mapped to a 2010 SOC category at the desired level of aggregation. Second, each four-digit industry was either mapped to the essential workers data, in which case the industry was deemed essential, or not, and the industry was classified as non-essential.
At this point we had employment, mean wage, and a binary indicator for essential industries, at the occupation-industry level. We then computed employment-weighted means to aggregate over all industries and estimate the share of each occupation in essential industries.

\section{Teleworkable, O*NET-SOC score at the 3-digit level, from Dingel and Neiman}

\subsection{Data}
Using a binary indicator of whether an occupation is \emph{teleworkable} from \cite{DN}, we aggregate up to 2010 SOC minor occupation categories.

\subsection{Methods}
Note that for some six-digit occupations, the teleworkable indicator was not provided. In these cases, the occupations with a missing indicator were ignored and we proceeded with aggregation.

We aggregated upwards starting from the finest categories possible. We took employment-weighted averages when each of the occupations listed in Dingel-Neiman had a non-missing employment estimate in OES for the given sector. Elsewhere, we took arithmetic means. 

This means that, for example, once we averaged out the SOC-O*NET occupations to 6-digit SOC codes, we might run into a case where two of the three 6-digit occupations in Dingel-Neiman for a given 5-digit occupation category have nonmissing employment estimates in OES, but one is missing in OES. In that case, we took another arithmetic mean over the teleworkable variable for these three occupations to aggregate to the 5-digit level. From there, we would then do the same thing: if all occupations in SOC at the given 5-digit category had nonmissing employment estimates in OES, we would use an employment-weighted mean of teleworkable at the current aggregation level to aggregate upward, and if not, we would use an arithmetic mean, etc...

\section{Teleworkable, manual score at the 5-digit level, from Dingel and Neiman}

\subsection{Data}
We start with a teleworkable indicator taking the values of 0, 0.25, 0.5, 0.75, and 1 from \cite{DN}, available at the 5-digit 2010 SOC level. We rounded values of 0.25 down to zero and values of 0.75 up to one. Values of 0.5 were changed to zero or one by discretion and the recoded occupations are shown in Table \ref{teleworkable_recode}.


\begin{table}
\begin{tabular}{lrr}
\toprule
{} &  original &  recode \\
Occupation                                          &               &         \\
\midrule
Social and Community Service Managers              &           0.5 &       0 \\
Claims Adjusters, Appraisers, Examiners, and In... &           0.5 &       1 \\
Appraisers and Assessors of Real Estate            &           0.5 &       0 \\
Marine Engineers and Naval Architects              &           0.5 &       1 \\
Agricultural and Food Science Technicians          &           0.5 &       0 \\
Counselors                                         &           0.5 &       0 \\
Social Workers                                     &           0.5 &       0 \\
Miscellaneous Community and Social Service Spec... &           0.5 &       0 \\
Clergy                                             &           0.5 &       0 \\
Audio-Visual and Multimedia Collections Special... &           0.5 &       0 \\
Farm and Home Management Advisors                  &           0.5 &       1 \\
Artists and Related Workers                        &           0.5 &       0 \\
Therapists                                         &           0.5 &       1 \\
Medical Records and Health Information Technicians &           0.5 &       0 \\
Miscellaneous Sales and Related Workers            &           0.5 &       1 \\
Miscellaneous Communications Equipment Operators   &           0.5 &       0 \\
Bill and Account Collectors                        &           0.5 &       0 \\
Payroll and Timekeeping Clerks                     &           0.5 &       0 \\
Procurement Clerks                                 &           0.5 &       0 \\
Miscellaneous Financial Clerks                     &           0.5 &       0 \\
Brokerage Clerks                                   &           0.5 &       0 \\
Correspondence Clerks                              &           0.5 &       0 \\
Loan Interviewers and Clerks                       &           0.5 &       1 \\
New Accounts Clerks                                &           0.5 &       1 \\
Dispatchers                                        &           0.5 &       0 \\
Office Clerks, General                             &           0.5 &       0 \\
Miscellaneous Office and Administrative Support... &           0.5 &       1 \\
\bottomrule
\end{tabular}
\caption{Recoded values of teleworkable variable by occupation.}
\label{teleworkable_recode}
\end{table}

\bibliography{bibliography}
\bibliographystyle{plain}

\end{document}