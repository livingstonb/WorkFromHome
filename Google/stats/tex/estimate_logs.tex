\documentclass{article}

\usepackage{amsmath}
\usepackage{booktabs}
\usepackage[margin=0.5in]{geometry}
\pagestyle{empty}

\usepackage{verbatim}

\title{Mobility and COVID-19 infection rate}
\author{}
\date{}

\begin{document}
\maketitle

\section{Model}
The model in levels is:
$$
\log m_{ct} = \alpha_1 I_{ct} ^ {\alpha_2} + \sum_j \beta_j P_{ct}^j + \epsilon_{ct}
$$
where $P_{ct}^j$ are the policy dummies.

\section{Heckman Correction}
Since observations are only included in the sample prior to or on the date of enactment of shelter-in-place, the potential exists for selection bias. First, I estimate the following probit model of the selection equation, where $\chi_{ct}$ is the shelter-in-place dummy recoded to be one when the county is not under shelter-in-place (is eligible to be included in the sample) and zero when shelter-in-place is active:
$$
Pr(P_{ct}^{SIP} = 0 \mid X_{ct})
= \gamma_0 + \gamma_1 t +
I_{ct} * (\gamma_2 + \gamma_3 rural_c + \gamma_4 republican_c + \gamma_5 icubeds_c + \gamma_6 popdensity_c) + \nu_{ct}
$$
I estimate this model using the window of 2/24 to 4/15.

Next, I take the fitted values and compute the inverse mills ratio, which is then included in the non-linear regression, and so the following is then estimated in the second stage:
$$
\log m_{ct} = \alpha_1 I_{ct} ^ {\alpha_2} + \sum_j \beta_j P_{ct}^j + \lambda(X_{ct}'\hat{\gamma}) + \epsilon_{ct}
$$

\end{document}